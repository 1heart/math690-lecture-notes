\documentclass[11pt]{article}
\usepackage{latexsym}
\usepackage{graphicx}
\usepackage{amsmath,amssymb,amsthm,calc}
\usepackage{euscript}
\usepackage{mathpazo}
\usepackage{mathtools}
\usepackage{tabularx}
\usepackage{bm}
\usepackage{makecell}


\usepackage[linesnumbered,lined,boxed,commentsnumbered]{algorithm2e}

\usepackage[top=1.25in,bottom=1.25in,left=2in,right=2in]{geometry}
\usepackage{enumerate}
\usepackage{subfig}
\usepackage{microtype}
\usepackage{fullpage}
\usepackage{xspace}
\usepackage{tikz,color}  % Used for the extended footnotes if needed
\definecolor{newblue}{rgb}{0.2,0.2,0.6}
\usepackage[colorlinks,allcolors=newblue]{hyperref}
\newcommand*{\vertbar}{\rule[-1ex]{0.5pt}{2.5ex}}
\newcommand*{\horzbar}{\rule[.5ex]{2.5ex}{0.5pt}}
\DeclareMathOperator{\E}{\mathbb{E}}

\newcommand{\handout}[5]{
   \renewcommand{\thepage}{#1-\arabic{page}}
   \noindent
   \begin{center}
   \framebox{
      \vbox{
    \hbox to 5.78in { {\bf MATH 690: Topics in Probablity Theory} \hfill #2 }
       \vspace{4mm}
       \hbox to 5.78in { {\Large \hfill #5  \hfill} }
       \vspace{2mm}
       \hbox to 5.78in { {\it #3 \hfill #4} }
      }
   }
   \end{center}
   \vspace*{4mm}
}

\renewcommand{\paragraph}[1]{\medskip \noindent {\bf #1.}}
%redefined math symbols taking arguments
\renewcommand\mod[1]{\ (\mathrm{mod}\ #1)}
\newtheorem{theorem}{Theorem}
\newtheorem{definition}{Definition}
\newtheorem{lemma}{Lemma}
\newtheorem{remark}{Remark}
\newtheorem{proposition}{Proposition}
\newtheorem{corollary}{Corollary}
\newtheorem{note}{Note}


\newcommand{\lecture}[4]{\handout{#1}{#2}{Lecturer: #3}{Scribe: #4}{#1}}
\DeclarePairedDelimiter\floor{\lfloor}{\rfloor}
\newcommand{\norm}[1]{\left\lVert#1\right\rVert}


\theoremstyle{definition}
\newcommand\PP{\mathbb{P}}
\newcommand\EE{\mathbb{E}}
\newcommand{\example}[1]{\paragraph{Example #1}}

\begin{document}

\lecture{Concentration of Measure}{October 26, 2017}{Xiuyuan Cheng}{Xiangying Huang}
\section{Introduction}
It is known from the central limit theorem (CLT) that given $X_1,X_2,\dots,X_n i.i.d$ with $\EE X_i=0$ and $Var(X_i)=\sigma^2$, we have
$$\frac{\sum_{i=1}^n X_i}{\sigma\sqrt{n}} \Rightarrow \mathcal{N}(0,1) \text{ in distribution.}$$
Write $S_n=\sum_{i=1}^n X_i$, $S_n \sim O(\sqrt{n})$. 
\begin{remark}
Notice that if $X_1,X_2,\dots,X_n$ are not mutually independent, then $S_n$ does not necessarily scale like $O(\sqrt{n})$. For example, when $X_1=X_2=\dots=X_n$, $S_n\sim O(n)$.
\end{remark}

While CLT ensures the convergence of $S_n$ in distribution, it does not give information on the rate of convergence, which is the main topic of this lecture. In terms of $S_n$, we want a result of the form
$$\PP(|S_n-\EE S_n|> \alpha)\leq ?$$
where $\alpha \sim O(\sqrt{n})$.

More generally, given a function $F(X_1,X_2,\dots,X_n)$, we want to bound the probability 
$$\PP(|F(X_1,X_2,\dots,X_n)-\EE F(X_1,X_2,\dots,X_n)|>\alpha)\leq ?$$

To motivate the study of concentration of measure, we list three applications of it:

\begin{enumerate}
\item \textbf{Johnson-Lindenstrauss Lemma}
\begin{lemma}Johnson-Lindenstrauss Lemma.

Given a set $X$ of $n$ points in $\mathbb{R}^D$, $0<\varepsilon<1$, and a number $\displaystyle d>\frac{8\ln n}{\varepsilon^2}$, there exists a linear map $f:\mathbb{R}^D\to \mathbb{R}^d$ such that
$$(1-\varepsilon)\norm{u-v}\leq \norm{f(u)-f(v)}\leq (1+\varepsilon)\norm{u-v}$$
for all $u,v\in X$.
\end{lemma}

\begin{remark}
$f\cdot u=\sqrt{\frac{D}{d}}P\cdot u$, where $P$ is a projection to a $d$-dimensional subspace and $u\in \mathbb{R}^D$.
\end{remark}

See \cite{johnson} for a proof.

\item \textbf{Fast randomized SVD}

Let $G$ be a $n\times k$ matrix where $k<<n$, $G{ij}\sim \mathcal{N}(0,1) \hspace{1ex}i.i.d$ and $G$ is well-conditioned. Let $u,v$ be different columns of the normalized matrix $\frac{G}{\sqrt{n}}$. We have
\begin{enumerate}
\item $\EE\norm{u}^2=1$
\item $\EE u^Tv=0$
\end{enumerate} 
Hence, $\frac{G}{\sqrt{n}}$ has ``almost" orthonormal columns and singular values close to 1. 

\item \textbf{Spectral norm of Wigner matrices}
\end{enumerate}

\section{Preliminaries}
\begin{lemma} Markov Inequality

Let $X$ be a non-negative random variable with $\EE X<\infty$. Then, for $\alpha>0$,
$$\PP(X>\alpha)\leq \frac{\EE X}{\alpha}$$
\end{lemma}

\begin{proof}
$\EE X = \EE [X(1_{X>\alpha}+1_{X\leq\alpha})]\geq \EE[X(1_{X>\alpha}]\geq \EE [\alpha1_{X>\alpha}]=\alpha\PP(X>\alpha)$
\end{proof}
\begin{corollary}Chebyshev's Inequality.
$$\PP(|X-EX|>\alpha)\leq \frac{Var(X)}{\alpha^2}$$
\end{corollary}


\section{Key Idea}\label{key}
Our goal is control the large deviation of $S_n$. Let us first consider the most basic case, where $X_1,X_2,\dots,X_n$ are $i.i.d$. (Constraints on identical distribution or independence can be relaxed.)

For $t>0$,
$$\PP(S_n>\alpha)=\PP(e^{tS_n}>e^{t\alpha})\leq \frac{E(exp(tS_n))}{e^{t\alpha}}.$$
Hence it suffices to give an upper bound of $E(exp(tS_n))$. By $i.i.d$ property of $\{X_i\}$,
\begin{align*}
\EE[exp(tS_n)]
&=\EE[exp(t\sum_{i=1}^n X_i)]\\
&=\prod_{i=1}^n \EE e^{tX_i}\\
&=( \EE e^{tX_1})^n
\end{align*}

If we assume $\EE e^{tX_1}\leq e^{ct^2}$ for some constant $c>0$ (which holds under rather general conditions),  the upper bound is then obtained:
$$\PP(S_n>\alpha)\leq  \frac{E(exp(tS_n))}{e^{t\alpha}} \leq \frac{e^{ct^2n}}{e^{\alpha t}}=e^{cnt^2-\alpha t}$$
Since the above inequality holds for any $t>0$, we can pick  $t^*=\frac{\alpha}{2cn}$ so that it minimizes the expression on the right hand side of the inequality. Plugging in $t^*$ gives
$$\displaystyle\PP(S_n>\alpha)\leq e^{\frac{-\alpha}{4cn}}$$
Notice that we pick $\alpha\sim O(\sqrt{n})$ so that the above bound gives meaningful results.

\section{Hoeffding's Inequality}
The idea in Section \ref{key} can be applied in the proof of similar results. One example is \textbf{Hoeffding's inequality}.
 
\begin{theorem} Hoeffding's Inequality.

Suppose $X_1,X_2,\dots,X_n$ are independent random variables with $a_i\leq X_i\leq b_i$. Write $S_n=\sum_{i=1}^n X_i$. Then, for $\alpha>0$,
$$P(|S_n-\EE S_n|>\alpha)\leq 2 exp(\frac{-2\alpha^2}{\sum_{i=1}^n(b_i-a_i)^2})$$
\end{theorem}
In order to prove the inequality, let us first introduce Hoeffding's lemma.
\begin{lemma}Hoeffding's lemma.

Suppose $X$ is a random variable such that $a\leq X\leq b$ for some constant $a,b$. For any $t>0$, we have
$$\EE e^{tX}\leq exp(\frac{t^2(b-a)^2}{8}).$$
\end{lemma}
\noindent\textbf{Proof of Hoeffding's Inequality:}
\begin{proof}
Without loss of generality we can consider the case where each $X_i$ has $\EE X_i=0$. Otherwise we just replace $X_i$ with $\bar{X}_i=X_i-\EE X_i$ in later analysis. Notice that if $a_i\leq X_i \leq b_i$, then $a_i-\EE X_i \leq \bar{X}_i\leq b_i-\EE X_i$, and  $(b_i-\EE X_i)-(a_i-\EE X_i )=b_i-a_i$. Hence the result remains the same.

With the above being said, from now on we work with $X_i$ that has $\EE X_i=0$. 
\begin{align*}
\PP(S_n>\alpha)&=\PP(e^{tS_n}>e^{t\alpha})\\
&\leq \frac{\EE e^{tS_n}}{e^{t\alpha}}\\
&=\frac{\prod_{i=1}^n \EE e^{tX_i}}{e^{t\alpha}} \\
&\leq e^{\frac{t^2\sum_{i=1}^n(b_i-a_i)^2}{8}}e^{-t\alpha} 
\end{align*}
By picking $t=\frac{4\alpha}{\sum_{i=1}^n(b_i-a_i)^2}$ we can minimize the right hand side of the inequality, which gives
$$\PP(S_n>\alpha)\leq  exp(\frac{-2\alpha^2}{\sum_{i=1}^n(b_i-a_i)^2}).$$

Following the same steps, we have 
$$\PP(S_n<-\alpha)=P(-S_n>\alpha)\leq  exp(\frac{-2\alpha^2}{\sum_{i=1}^n(b_i-a_i)^2}).$$
Therefore,
$$P(|S_n-\EE S_n|>\alpha)\leq 2 exp(\frac{-2\alpha^2}{\sum_{i=1}^n(b_i-a_i)^2}).$$
\end{proof}

\noindent\textbf{Proof of Hoeffding's lemma:}
\begin{proof}
By convexity, for any $x\in(0,1)$ we have 
$$e^{tX}\leq xe^{tb}+(1-x)e^{ta}.$$
Hence,
$$e^{tX}\leq \frac{X-a}{b-a}e^{tb}+\frac{b-X}{b-a}e^{ta}.$$
Then
$$\EE e^{tX}\leq \frac{\EE X-a}{b-a}e^{tb}+\frac{b-\EE X}{b-a}e^{ta}=\frac{-a}{b-a}e^{tb}+\frac{b}{b-a}e^ta.$$
Let $\displaystyle h=t(b-a), \displaystyle p={\frac {-a}{b-a}}$ and $\displaystyle L(h)=-hp+\ln(1-p+pe^{h})$. Then $\frac{-a}{b-a}e^{tb}+\frac{b}{b-a}e^ta=e^{L(h)}$.

Taking derivative of $\displaystyle L(h)$, 

$\displaystyle L(0)=L^{'}(0)=0$  and 
$$L^{''}(h)=\frac{(1-p)pe^h}{(1-p+pe^h)^2}\leq \frac{(1-p)pe^h}{4(1-p)pe^h}=\frac {1}{4}$$
 for all $h$.

By Taylor's expansion,

$$\displaystyle L(h)\leq \frac {1}{8}h^{2}=\frac {1}{8}t^{2}(b-a)^{2}$$

Hence, $\displaystyle \mathbb {E} e^{t X}\leq e^{{\frac {1}{8}}t^{2}(b-a)^{2}}$
\end{proof}








\bibliographystyle{alpha}
\bibliography{bib}
\begin{thebibliography}{[AHU]}
\bibitem{johnson} Sanjoy Dasgupta and Anupam Gupta. "An elementary proof of a theorem of Johnson and Lindenstrauss." Random Structures \& Algorithms 22.1 (2003): 60-65.
\end{thebibliography}

\end{document}
