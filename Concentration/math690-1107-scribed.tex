\documentclass{article}
\usepackage[utf8]{inputenc}
\usepackage{float}
\usepackage{color}
\usepackage{amsmath,amssymb}
\DeclareMathOperator{\E}{\mathbb{E}}
\DeclareMathOperator{\R}{\mathbb{R}}

\newtheorem{theorem}{Theorem}[section]
\newtheorem{corollary}{Corollary}[theorem]
\newtheorem{lemma}[theorem]{Lemma}

\title{Math690:Concentration(11/7/2017)}
\author{Guangshen Ma}
\date{November 2017}

\usepackage{natbib}
\usepackage{graphicx}

\begin{document}

\maketitle

\subsection{Review of last class}
Last time we discuss the concentration problem of $F(x)$, where it satisfies:
\[
F:~R^n~\rightarrow~R 
\]
where $x_1,x_2,{\cdots},x_n$ are independent.
\\
Today's topic focus on the concentration of spectral edge (The largest or smallest eigenvalues), which includes the following sections:
\begin{itemize}
    \item Widger matrix:~The reason why the norm of the Widger matrixes is concentrated 
    \item $\lambda_2$ of Erdos-Renyi random graph theory
    \item Randomized fast SVD(PCA)
\end{itemize}
\begin{lemma}
 $\lambda_1(A)~(The~largest    ~eigenvalue)~is~a~"1-lipchitz"~function$
\end{lemma}
\begin{itemize}
\item What the expectation of the spectral edge look like
\item The fluctuation of the these eigenvalues (measure by high probability) would not be far away from its expectation (Concentration)
\end{itemize}
\noindent
\textbf{Widger's theorem:} 
\[
W_{mxn},~W^T = W,~W_{ij}{\sim}N(0,1),~iid,~i{\leq}j,~(\lambda_1{\geq\lambda_2\geq{\cdots}\lambda_n~of~W)}
\] 
\noindent
\textbf{Wigner's semi-circle law:}\\
\noindent
Spectral distribution of $\frac{1}{\sqrt{n}}W~{\rightarrow}~\rho_{s.c}~(Weakly~converge)$.\\

\noindent
\textbf{TW(Tracy-Widom distribution) law:}\\
Spectral edge $\lambda_1$(similarly to $\lambda_n$) concentrates at $2$~(-2 for $\lambda_n$).
\noindent
Based on the previous review, our goal is to show that:
\[
\lambda_1(W) = 2\sqrt{n}+O(n)
\]
We want to show that high order term $O(n)$ is going to be $O(1)$ with high probability.
\\
\noindent 
Compared with the Tracy-widom result and we get:
\[
\lambda_1 = 2\sqrt{n} + O(n^{-\frac{1}{6}})
\]
This is like the true scaling of the concentration. But we are going to see our argument by Gaussian concentration equality to show the term $O(1)$ and how we control the deviation.

\noindent
To show the above results, we have two steps:\\
    \textbf{Step 1}: Show the expectation of $\lambda_1$
\[
\mathop{\mathbb{E}}\lambda_1(W) \approx 2\sqrt{n}
\]
    \textbf{Step 2}: Use $1-Lipchitz$ lemma to show the concentration
\[
\lambda_1~is~1-Lipchitz~function:~\R^{\frac{n(n-1)}{2}}{\rightarrow}\R
\]

\subsection{The first step}
We start by using some theory from References.\\

\noindent
\textbf{Proposition[Ref[1],Tao]}:
\[
\E \lambda_1 {\leq} 2\sqrt{n}
\]

\noindent
\textbf{Proof(Moment method[Ref[2]]}:\\
\noindent
We first consider the eigenvalues $\lambda_i$ as the ith sample and take the moment of each sample and we get: 
\[
\sum_{i=1}^n \lambda_i^k = Tr(W^k)
\]
This operation here is just the tool for the proof. Since we want to prove the upper bound and we relate $\lambda_1$ and get the upper bound argument(k can be considered as a very large even number):
\[
\lambda_1^{k} {\leq} \sum_{i=1}^n \lambda_i^k=Tr(W^k),~(k>0)
\]
Take expectation of both sides and we get:
\[
\E \lambda_1^k {\leq} \E Tr(W^k)
\]
By Jensen's inequality:
\[
(\E \lambda_1)^k {\leq} (\lambda^k)~\Rightarrow~(\E \lambda_1)^k{\leq}Tr(W^k)
\]
Recommend to take $k=2,4(or~higher~number)$ as an example for understanding the proof.\\
\noindent
We use another $Lemma$ without showing the details:
\[
\E Tr(W^k) = (C_{\frac{k}{2}}+O_k(1))n^{\frac{k}{2}+1}
\]
Where $k>0$ and k is fixed here.\\
\noindent
We also know(Catolan number): 
\[
C_{\frac{k}{2}} = \frac{k!}{(\frac{k}{2}+1)!(\frac{k}{2})!}
\] 
If we take $(C_{\frac{k}{2}})^{\frac{1}{k}}{\rightarrow}2+O(1)$ when $k{\rightarrow}\infty$(This is true without providing details here). That is the reason we have the number $2$ for the upper bound in the inequality.\\
\noindent
Take both sides multiplied by order $\frac{1}{k}$ and we get:
\[
\E Tr(W^k)^{\frac{1}{k}} = [( C_{\frac{k}{2}} + O_k(1) )n^{\frac{k}{2}+1}]^{\frac{1}{k}}
\]
The right hand side will be:
\[
[C_{\frac{k}{2}} + O_k(1)]^{\frac{1}{k}}{\cdot}n^{\frac{1}{2}+\frac{1}{k}}
\]
As $k \rightarrow \infty$ and we get:
\[
\E \lambda_1 {\leq} (2+O(1)){\cdot}\sqrt{n}
\]
In practice, we usually take $k{\rightarrow}(lg^n)^2$(missing details here) and we have:
\[
\E \lambda_1 {\leq} (2+\epsilon)\sqrt{n}
\]
This argument is corresponding to our purpose, i.e control $\lambda_1$ and its expectation value. That also explains why the semi-circle law we have the number 2.\\

\noindent
Lower bound(similar argument with moment method):
\[
Tr(W^k) = \sum_i \lambda_i^k~{\leq}~n{\cdot}\lambda_1^k
\]
Similarly, take expectation on both sides and we can show that:
\[
\E \lambda_1 {\geq} (2-\epsilon)\sqrt{n}
\]
So far we have finished the step 1.  

\subsection{The second step}
One lipschitz function problem:
\begin{lemma}
$|\lambda_1(A) - \lambda_1(B)| {\leq} ||A-B||_{Fros}$
\end{lemma}
Where $\lambda_1~is~a~one-lipschitz~function~of~matrix~A$). $A_{nxn}$ and $B_{nxn}$ are symmetric. We also have $\lambda_1:\R ^{\frac{n(n-1)}{2}}\rightarrow \R$ is $1-Lipschitz$. The $1-Lipschitz$ mentioned here means the lipschitz constant C here is 1.\\

\noindent
\textbf{Proof of Lemma(Courant-Fisher):}
\[
|\lambda_1(A) - \lambda_1(B)|{\leq}||A-B||_{op}
\]
Where $op$ symbol means operator norm.
\[
\lambda_1(A) = \mathop{sup}\limits_{||v||_2 =1}v^TAv
\]
\[
\lambda_1(B) = v_0^TBv_0 = v_0^TAv_0 + v_0^T(B-A)v_0
\]
Where the first term $v_0^TAv_0<\lambda_1(A)$.\\

\noindent
\textbf{Remark(Operator norm is bounded by Frobos norm):}
\[
\sum_{i=1}^{n}|\lambda_i(A) - \lambda_i(B)|^2 {\leq} ||A-B||_{Fro}^2
\]
Then we have(Citing the results from reference):
\[
|\lambda_i(A) - \lambda_i(B)|^2 {\leq} ||A-B||_{Fro}^2
\]
We use the above propositions to prove the lemma in step 2, and for $i=1$ we have:
\[
Pr[|\lambda_1 - \E \lambda_1|>\alpha]~{\leq}~2{\cdot}e^{-\frac{\alpha^2}{2}}
\]
Therefore, the probability of the deviation from $\lambda_1$ and its expectation is bounded and also decay exponentially(The concept of concentration).

\subsection{The Cheager's constant number}
\noindent
\textbf{Remark:}\\
\noindent
$\lambda_2$ is the second eigenvalue of the normalized graph Laplacian.
\[
L_{rw} = D^{-1}(D-W) = I - P
\] 
The symbol $rw$ means random walk laplacian. And we have:
\[
0 = \lambda_1<\lambda_2{\leq}{\cdots}{\leq}\lambda_n({\leq}2)
\]
The $\lambda_2$ here is strongly related to how much you can cut your graph.\\

\noindent
\textbf{Remark:~Cheager's law}
\[
G = (V,E),~W_{nxn},~W_{ij}{\geq}0,~W = W^T
\]
Define:
\[
h(A) = \frac{|\partial A|}{min \{Vol(A),Vol(A^c) \} }
\]
Where:
\[
(\partial A) \mathop{\Longrightarrow}\limits^{def} \sum_{i{\in}A,j{\in}A^c}W_{ij}
\]
\[
Vol(A) = \sum_{i{\in}A}d_i,~d_i\mathop{\Longrightarrow}\limits^{def}\sum_{j=1}^n W_{ij}
\]
We define Cheager's constant as $h_G = \mathop{min}\limits_{A\subset v} h(A)$. It measures how cutting a graph or manifold it is.\\

\noindent
\textbf{Theorem:~Cheager Inequality}
\[
\frac{1}{2}h_G^2{\leq}\lambda_2{\leq}2h_G
\]
The hint of proving $\lambda_2 {\leq} 2h_G$:
\[
\lambda_2 = \mathop{inf}\limits_f \frac{f^TLf}{f^TDf}{\leq}\frac{g^TLg}{g^TDg},~L = D- W
\]
Where g is:
%$$ g=\left\{
%\begin{aligned}
%\alpha~on~A \\
%\beta~on~A^c
%\end{aligned}
%$$
$\lambda_2$ gives implication on how cutting it is for the graph.
\subsection{Application with fast PCA}
Key ideas:
\begin{itemize}
\item PCA:~We want to compute the low rank factorization of $A_{nxn}$ 
\item Compute the first k singular values \\
\end{itemize}
Key idea:~Tall gaussian matrix is well-conditioned
\[
x_{ij} \sim N(0,1),~iid~(k{\ll}n)
\]
We have the demo showing the follow properties:
\[
\sigma_1(x){\geq}{\cdots}{\geq}\sigma_k(x)
\]
\[
\sqrt{n} -\sqrt{k}-O(1){\leq}\sigma_k(x){\leq}\sigma_1(x){\leq}\sqrt{n}+\sqrt{n}+O(1)
\] 
Based on the MP(mention in 9/7's note)density:
\begin{equation}
\left\{
\begin{aligned}
\sigma_1^2 {\approx} (1+\sqrt{\frac{k}{n}})^2 \\
\sigma_k^2 {\approx} (1-\sqrt{\frac{n}{k}})^2 \\
\end{aligned}
\right.
\end{equation}

\subsection{Appendix}
\begin{itemize}
\item Lipschitz continuty
\item Courant-Fischer Minimax Theorem(9/5's note)
\item $L_{rw}$ random walk laplacian matrix 
\item Jensen's inequality
\item Wigner's Semicircle Law
\item Tracy–Widom distribution
\item Method of moments
\item MP law(9/7's note)
\end{itemize}



\end{document}


\bibliography{references}