\documentclass[english]{article}

\usepackage[a4paper,margin=2cm]{geometry}
\usepackage{setspace}
\onehalfspacing

\usepackage{amsmath}
\usepackage{amsthm}
\usepackage{amssymb}

\usepackage{babel}

\usepackage{mathptmx}
\usepackage{amsmath}
\usepackage{graphicx}
\usepackage{color}

\newcommand{\Tr}{{\bf Tr}}
\newcommand{\R}{\mathbb{R}}
\newcommand{\E}{\mathbb{E}}
\newcommand{\N}{\mathcal{N}}
%\newcommand{\Pr}{ \text{Pr} }


\newtheorem{theorem}{Theorem}[section]
\newtheorem{conject}{Conjecture}[section]

\newtheorem{lemma}{Lemma}[section]
\newtheorem{corollary}{Corollary}[section]
\newtheorem{proposition}{Proposition}[section]


%% Choose one of the following (if not choosing the  
%% default, viz., Computer Modern, font family):
 %\usepackage{lmodern}
 %%
 %\usepackage{mathpazo}
% \usepackage[theoremfont]{newpxmath} \usepackage{newpxmath}
 %\usepackage{kpfonts}
 %%
 %\usepackage{mathptmx}
 %\usepackage{times,mtpro2}
 %\usepackage{stix}
 %\usepackage{txfonts}
 %\usepackage{newtxtext,newtxmath}
 %%
 
 %\usepackage{libertine} \usepackage[libertine]{newtxmath}
 
 %\usepackage{newpxtext} \usepackage[euler-digits]{eulervm}


\begin{document}

\title{Math 690 F2017: Topics in Data Analysis and Computation\\
Homework 6}

\author{Xiuyuan Cheng}
\date{}

\maketitle

\begin{enumerate}

\item

Using concentration argument, finish the proof of the upper bound of the Johnson-Lindenstrauss lemma.
The proof is given in [DG03].

\item

Study the concentration of $\lambda_2$, the second smallest eigenvalue, of the normalized graph laplacian of an Erdos-Renyi random graph $G(n,p)$, i.e. the graph has $n$ nodes and the probability of $A_{ij}=1$ equals $p \in (0,1)$. We know that when the graph is connected (which happens almost surely if $p > (1+\epsilon)\frac{ \log n }{n}$, as proved in the classical work of Erdos and Renyi in 1960), $0< \lambda_2 < 2$.

(1) Let $p$ be fixed constant, and $n$ increases. What is the limiting statistics of $\lambda_2$ like? (Hint: $\lambda_2 \to a$ for some constant $a$, and after properly centering and rescaling $(\lambda_2 - a)$ converges to a limiting distribution.)

(2) What happens if $p$ decreases with $n$, e.g. $p = \alpha \frac{\log n }{n }$ for $\alpha > 1$? Numerically observe the distribution of $\lambda_2$ in this case.

\end{enumerate}

\end{document}
