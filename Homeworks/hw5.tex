\documentclass[english]{article}

\usepackage[a4paper,margin=2cm]{geometry}
\usepackage{setspace}
\onehalfspacing

\usepackage{amsmath}
\usepackage{amsthm}
\usepackage{amssymb}

\usepackage{babel}

\usepackage{mathptmx}
\usepackage{amsmath}
\usepackage{graphicx}
\usepackage{color}

\newcommand{\Tr}{{\bf Tr}}
\newcommand{\R}{\mathbb{R}}
\newcommand{\E}{\mathbb{E}}
\newcommand{\N}{\mathcal{N}}
%\newcommand{\Pr}{ \text{Pr} }


\newtheorem{theorem}{Theorem}[section]
\newtheorem{conject}{Conjecture}[section]

\newtheorem{lemma}{Lemma}[section]
\newtheorem{corollary}{Corollary}[section]
\newtheorem{proposition}{Proposition}[section]


%% Choose one of the following (if not choosing the  
%% default, viz., Computer Modern, font family):
 %\usepackage{lmodern}
 %%
 %\usepackage{mathpazo}
% \usepackage[theoremfont]{newpxmath} \usepackage{newpxmath}
 %\usepackage{kpfonts}
 %%
 %\usepackage{mathptmx}
 %\usepackage{times,mtpro2}
 %\usepackage{stix}
 %\usepackage{txfonts}
 %\usepackage{newtxtext,newtxmath}
 %%
 
 %\usepackage{libertine} \usepackage[libertine]{newtxmath}
 
 %\usepackage{newpxtext} \usepackage[euler-digits]{eulervm}


\begin{document}

\title{Math 690 F2017: Topics in Data Analysis and Computation\\
Homework 5}

\author{Xiuyuan Cheng}
\date{}

\maketitle

\begin{enumerate}

\item

(Non-local means)
Apply non-local means to the problem of (patch based) image denoising, and consider the following modifications

(i) Self-tuning of the $\sigma$ in heat kernel: let 
\[
W_{ij} = \exp \{ -\frac{ \| x_i - x_j\|^2}{ 2 \sigma_i \sigma_j}\},
\]
where $\sigma_i$ equals the distance of the $k$-the nearest neighbor of data point $x_i$.  The parameter $k$ is set to be a constant, e.g. $k=10$. 

(ii) The combination of both local and non-local means: let 
\[
W_{ij} = \exp \{ -\frac{ \| x_i - x_j\|^2}{ 2 \sigma_i \sigma_j}\}
              \cdot   \exp \{ -\frac{ \| u_i - u_j\|^2}{ 2 \sigma^2}\}
\]
where $u_i$ stands for the position of the center pixel of the patch on the 2D grid.

What are the effects of these modifications? Compare the performance with e.g. PCA.

\item

($\mathbb{Z}_2$ synchronization) 
The problem is to recover some arbitrary signs $\{z_i\}_{i=1}^n$ on $n$ nodes, $z_i = \pm 1$, from $\frac{n(n-1)}{2}$ noise-corrupted observations: when $i<j$,
\[
G_{ij} = \begin{cases}
 z_iz_j, & \quad \text{ with prob. $p$},\\
 W_{ij}, & \quad \text{ with prob. $1-p$},
\end{cases}
\]
and $G_{ij}=G_{ji}$, $G_{ii} =1$, and $W$ is a symmetric matrix consisting of i.i.d. random signs, i.e. $W_{ij} = \pm 1 $ with prob. $\frac{1}{2}$.
Apply spectral methods, i.e. using the eigenvector associated with the largest eigenvalue, 
to the problem, and study how the successful sign recovery depends on the correct-observation rate $p$ and the number of nodes $n$.



\end{enumerate}

\end{document}
