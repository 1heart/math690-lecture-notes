\documentclass[12pt]{article}

\usepackage[a4paper,margin=2cm]{geometry}
\usepackage{amsmath, amssymb, amsthm, amsfonts, tikz, algpseudocode}
\usepackage[plain]{algorithm}
\usepackage[framemethod=default]{mdframed}
\usepackage{dsfont}

\theoremstyle{plain}
\newtheorem*{theorem}{Theorem}
\newtheorem*{lemma}{Lemma}
\newtheorem*{claim}{Claim}
\newtheorem*{definition}{Definition}
\newtheorem*{corollary}{Corollary}
\newtheorem*{remark}{Remark}
\newtheorem*{proposition}{Proposition}
\DeclareMathOperator*{\argmin}{arg\,min}
\DeclareMathOperator*{\eig}{eig}
\DeclareMathOperator*{\diag}{diag}
\DeclareMathOperator*{\vol}{Vol}

%%%% TITLE

\title{Math 690: Topics in Data Analysis and Computation \\
Lecture notes for October 12, 2017}
\date{}

\author{Scribed by Dev Dabke and Andrew Cho}

\begin{document}
\maketitle

\section{Introduction}
The lecture covered the following on the $\textbf{consistency of spectral clustering}$:
\begin{itemize}
	\item Setup of $\mathcal{L}_{un}$ and $\mathcal{L}_n$ and their limit operators $U$ and $U'$
	\item First $r$ spectral convergence
	\item Bochner's Theorem
\end{itemize}

\section{Consistency of Spectral Clustering}

\subsection{The Problem Setup}

Suppose $ X_{i} \sim P $ where $ P $ is some distribution on $ \Omega \subset \mathbb{R}^{D} $.

$ W_{ij} $ is the affinity matrix where, as an example,
\[
W_{ij} = e^{\frac{-|x_i - x_j|^2}{\varepsilon}}
\]
where $ \varepsilon > 0 $.
\\ \\
As $ n \to \infty $, we want to show the convergence of the graph Laplacian $ \mathcal{L} $.
\begin{enumerate}
  \item $ \mathcal{L}_{n} = D - W $ where $ D_{ij} = \sum_{j = 1}^{n} W_{ij} $ ($  \to U $), i.e.\ unnormalized
  \item $ \mathcal{L}_{n}' = D^{-\frac{1}{2}} (D - W) D^{-\frac{1}{2}} $ where ($  \to U' $), i.e.\ symmetric
  \item $ \mathcal{L}_{n}'' = D^{-1} (D - W) = I - P $, i.e.\ random walk
\end{enumerate}

\subsection{Limit operators $U$ and $U'$}
We construct linear limit operators $U$ and $U'$ on $C(X)$ which are the limit of the discrete operators $\mathcal{L}_n$ and $\mathcal{L}_n'$. We prove that the first "r" eigenvectors of the discrete operators converge to eigenfuction  of the limit operators. 
\\ \\
Here, $U$ is defined as follows:
\begin{align*}
  U &: C(\Omega) \to C(\Omega) \\
  Uf(x) &= f(x)d(x) - \int k(x, y) f(y) dP(y)
\end{align*}
where
\begin{align*}
  dP(x) &= p(x)dx \\
  dx &= \int k(x, y) dP(y) \\
  x \in \Omega
\end{align*}

$U'$ is derived and proved as follows:
\[
U' f(x) = f(x) - \int{ \frac{ k(x, y) }{ \sqrt{ d(x) } \sqrt{ d(y) } } f(y) dP(y) }
\]

\begin{proof}

  \begin{align*}
    (D^{-\frac{1}{2}} W D^{-\frac{1}{2}})_{ij} &= \frac{1}{\sqrt{D_{ii}}} W_{ij} \frac{1}{\sqrt{D_{jj}}} \\
    &= \frac{ \frac{1}{n} k(x_i, x_j) }{ \sqrt{ \frac{1}{n} \sum_{j'} k(x_i, x_j') } \sqrt{ \frac{1}{n} \sum_{j'} k(x_j, x_j') } } \\
    &\approx \frac{ \frac{1}{n} k(x_i, x_j) }{ \sqrt{d(x_i)} \sqrt{d(x_j)} } \\
  \end{align*}

\end{proof}

\subsection{First $ r $ spectral convergence}

Let's discuss what "first $r$ spectral convergence means. $ M_n $ first $r$ spectral convergence to T if first r eigenvalues of $ M_n $ converge to those of T, and the associated eigenvectors converge to the eigenfunctions of T, the first smallest r eigenvalues.

\begin{theorem}
For fixed $r > 0$, $n \rightarrow \infty$, and mild conditions,
	\begin{enumerate}
	\item ($\textbf{unnormalized}$) $\mathcal{L}_n$ first r spectral converge to $ U $ if the first r eigenvalues of $ U $ lie outside of the range of the degree function $ d(x) $. We need extra constraints for convergence since $ U $ might coincide with the range of $ d(x) $. 	
	\item ($\textbf{normalized symmetric}$) $\mathcal{L}_n'$ first r spectral converge to the operator $ U' $. 
	\end{enumerate}
\end{theorem}

\begin{proof}
(Case 2 $\mathcal{L}_{sym}$) Have $M_n$ converge to $T$ where 	
	\begin{align*}
	&M_n: \mathds{R}^n \rightarrow \mathds{R}^n \\
	&T: C(\Omega) \rightarrow C(\Omega) \\
	&M_n: I-D^{\frac{-1}{2}}WD^{\frac{-1}{2}} \\
	&T = U'
	\end{align*}	

	\begin{align*}
	Tf(x) &= f(x) - \int h(x,y)f(y)dP(y) \\
	&= f(x) - \int h(x,y)f(y)dP_n(y) \\
	&= f(x) - \frac{1}{n} \sum_{i=1}^n h(x,x_i)f(x_i)
	\end{align*}
where
	\begin{align*}
	&h(x,y) = \frac{k(x,y)}{\sqrt{d(x)}\sqrt{d(y)}} \\
	&dP_n(x) = \frac{1}{n} \sum_{i=1}^n \delta_{x_i}(x) dx
	\end{align*} 
	
\end{proof}

$\textbf{Lemma 1}$ (spectral equivalence between $M_n$ and $T_n$)
\begin{enumerate}
\item If $T_n \varphi = \lambda \varphi$, let $v \in \mathds{R}^n, i = 1,...,n v_i = \varphi(x_i)$, then $M_nv = \lambda v$. 
\item If $M_nv = \lambda v$ and $\lambda \neq 1$, then let $\varphi = \frac{\frac{1}{n} \sum h(x,x_j)v_j}{1-\lambda}$ and so then $T_n\varphi = \lambda\varphi$.
\end{enumerate}

$\textbf{Lemma 2}$ (replacing $dP_n(y)$ to be $dP(y)$, $T_n$ first spectral converge to T)

\begin{proof}
For all $f$, $T_n \rightarrow Tf$ by Law of Large Numbers. $\|T_nf - Tf\|_{\inf} \rightarrow 0$ simultaneously for "sufficiently many" $f$ such that for each eigenvalue of $T (\lambda \neq 1)$, the associated eigenvalue of $T_n$ converge to $\lambda$ and the associated eigenfunction of $T_n$ converges. In other words, 
$ T_n \varphi_n = \lambda_n \varphi_n $ so $ \lambda_n \rightarrow$asymptotically 
$T\varphi = \lambda \varphi$ and $ \| \varphi_n- \varphi \|_{\infty} \rightarrow 0$
\end{proof}

$\textbf{Remark}$ (Bochner's Theorem) \\
$W$ is a PSD where $W_{ij}$ comes from Gaussian kernel $\frac{e^{-|x_i - x_j|^2}}{\epsilon}$. This implies $D^{-\frac{1}{2}}WD^{-\frac{1}{2}})$ is PSD. Thus, we have $\mathcal{L} = I - D^{-\frac{1}{2}}WD{-\frac{1}{2}},$ with eigenvalues on [0,1] since eigenvalues of $D^{-\frac{1}{2}}WD{-\frac{1}{2}} \in [0,1)$

\begin{proposition}
P is a density in $\mathds{R}^D$. We have the usual Gaussian kernel $\frac{e^{-|x_i - x_j|^2}}{\epsilon}$ where $\epsilon > 0$. 
	\begin{enumerate}
	\item k(x,y) is PSD kernel, ie. $\forall f$ and $\int f(x)^2p(x)dx < \infty$, $\int_\Omega \int_\Omega k(x,y)f(x)f(y) p(x) dx p(y) dy \geq 0$. 
	\item $W_{ij} = (k(x_i,x_j))_{1 \leq i, j \leq n}.$ We know W is PSD, so $\forall v \in \mathds{R}^n$, $v^TWv \geq 0$. 
	\end{enumerate}
\end{proposition} 

\begin{proof}
By Fourier Transform,
$$ k(x,y) = g(x-y) = \int_\mathds{R}^D \hat{g}(\xi)e^{i\xi^T(x-y)}d\xi $$

	\begin{align*}
	\forall v \in \mathds{R}^n, v^TWv = \ &=\sum_{ij}v_i v_j W_{ij} \\
	&= \sum_{ij} v_i v_j k(x_i,x_j) \\
	&= \sum v_i v_j \int \hat{g}(\xi)e^{i\xi^Tx_i}e^{-i\xi^Tx_j} d\xi \\
	&= \int_\mathds{R}^D \hat{g}(\xi)(\sum_i^n \sum_j^n v_i e^{i\xi^Tx_i} v_j e^{-i\xi^Tx_j} d\xi) \\
	&= \int_\mathds{R} \hat{g}(\xi)(\sum v_i e^{i\xi^Tx_I})\overline{(\sum v_j e^{i\xi^Tx_j}}) \\
	&= \int \hat{g}(\xi) V(\xi) \overline{V(\xi)} d\xi \\
	&= \int \hat{g}(\xi) |V(\xi)|^2 d\xi \geq 0
	\end{align*}
since $\hat{g}(\xi) = e^{-a|\xi|^2} \implies \hat{g} > 0,  \forall \xi$ 
\end{proof}


\end{document}
