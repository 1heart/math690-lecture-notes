\documentclass[12pt]{article}

\usepackage[a4paper,margin=2cm]{geometry}
\usepackage{amsmath, amssymb, amsthm, amsfonts, tikz, algpseudocode}
\usepackage[plain]{algorithm}
\usepackage[framemethod=default]{mdframed}

\theoremstyle{plain}
\newtheorem*{theorem}{Theorem}
\newtheorem*{lemma}{Lemma}
\newtheorem*{claim}{Claim}
\newtheorem*{definition}{Definition}
\newtheorem*{corollary}{Corollary}
\newtheorem*{remark}{Remark}
\newtheorem*{proposition}{Proposition}
\DeclareMathOperator*{\argmin}{arg\,min}
\DeclareMathOperator*{\eig}{eig}
\DeclareMathOperator*{\diag}{diag}
\DeclareMathOperator*{\vol}{Vol}

%%%% TITLE

\title{Math 690: Topics in Data Analysis and Computation \\
Lecture notes for October 5, 2017}
\date{}

\author{Scribed by Dev Dabke and Andrew Cho}

\begin{document}
\maketitle

\section{Background}

Recall $ W_{n \times n} $ where $ W = W^{\intercal} $ (i.e.\ symmetric) and $ w_{ij} > 0 $.
Additionally, let $ d = \diag{d_1, \ldots, d_n} $ where $ d_i = \sum_{j = 0}^{n} w_{ij} $.
Next, denote
\begin{align*}
  \mathcal{L}_{wn} &= D - W \\
  \mathcal{L}_{rw} &= I - D^{-1} W
\end{align*}
and $ P = D^{-1} W $.
\\ \\
Recall that if a graph has $ k $ connected components, then the eigenspace of $ \mathcal{L}_{wn} $ associated with eigenvalue $ 0 $ is the span of $ \{ \mathbf{1_{A_{i}}}, \ldots, \mathbf{1_{A_{i}}} \} $.

\section{Exploring $ \mathcal{L} $}

\subsection{Redefining Matrices}

What is $ \mathcal{L}_{rw} $?
We see that $ \mathcal{L}_{rw} = \Psi(1 - \Lambda)\Psi^{\intercal} $ and that $ P = \Psi\Lambda\Psi^{\intercal} $.
Next, we note that $ \Psi^{\intercal} D \Psi = I $.
Now, denoting $ \Psi = [ \varphi_1, \ldots, \varphi_{n} ] $, we can see that $ \varphi_k^{\intercal} D \varphi_l = 0 $ if $ k \neq l $.
Finally, let $ P \phi_k = \lambda_k \phi_l $.

\subsection{Eigenvectors}

If $ f $ is an eigenvector of $ \mathcal{L}_{wn} $ with $ \lambda = 0 $, $ \mathcal{L}_{wn} f = 0 \dot f $ and $ (D - W)f = 0 $.
If $ f $ is an eigenvector of $ \mathcal{L}_{rw} $ with $ \lambda = 0 $, then $ D^{-1} (D - W) f = 0 $ and $ (D - W)f = 0 $.
\\ \\
Next, we can also write the eigenvectors $ \varphi_{1} = c_1 \mathbf{1_{A_{1}}} + \cdots + c_k \mathbf{1_{A_{k}}} $.
\\ \\
What are the constants?
\[
c_1 = \sqrt{\frac{1}{\sum_{i \in C_1}d_i}}
\]
so $ c_1 = \{ 1, \ldots, n_1 \} $ and $ c_2 = \{ n_1 + 1, \ldots, n_1 + n_2 \} $.

\subsection{The Special Case of $ k = 2$}

We can write $ \widetilde{\varphi_2} = \alpha \varphi_{1}  + \beta \varphi_{2} $.
We have that $ \alpha^2 + \beta^2 = 1 $ and that $ \widetilde{\varphi_2}^{\intercal}d = 0 $ since $ d = [d_1, \ldots d_n]^{\intercal} $.
\\ \\
Moreover, $ \alpha $ and $ \beta $ should have different signs, which we can use to get clusters.
More generally, we don't have to cut at zero, but using the sign as a cluster is convenient.

\subsection{Deciding $ k $}

We can use the spectral gap heuristic, which tells us that the first $ k $ eigenvalues will be close to one another, and then there will be a ``gap'' after $ k $.
We could use the log values of the ordered eigenvalues can provide some estimate as to where this gap could be.

\section{Graph Cut}

\subsection{Getting Started}

\begin{definition}[Cut]

Let $ A \cap B = \emptyset $ where $ A, B \subset V = \{ 1, \ldots, n \} $.
Then a cut $ W(A, B) $ is defined as
\[
W(A, B) = \sum_{i \in A, j \in B} w_{ij}
\]

\end{definition}

\begin{definition}[Volume Set]

$ \vol(A) = \sum_{i \in A} d_i $ is the sum of degrees.

\end{definition}

\begin{definition}[Normalized Cut Partition]
Let $ C = \{ C_1, \ldots, C_k \} $ and denote
\[
NCut(C) = \frac{ \sum_{l = 1}^k W(C_1, C_l^{\complement}) }{ \vol{C_l} }
\]

\end{definition}

\begin{proposition}
  \begin{align*}
    NCut(C) &= \sum_{l = 1}^{k} \varphi_l^{\intercal} L \varphi_{l} \\
    \varphi_l &= c_{l} \mathbf{1_{c_{l}}} & l = 1, \ldots, k \\
    c_l &= \frac{1}{\sqrt{\vol{C_l}}}
  \end{align*}
\end{proposition}

\subsection{Problem of Min NCut}

Let $ H $ be the matrix such that $ H = [\varphi_1, \ldots, \varphi_k] $.
Then, the minimum $ NCut $ over the $ C $s is the trace of $ H^{\intercal} L H $.
\\ \\
The spectral minimum $ NCut $ is defined to be the above trace, but where $ H^{\intercal} D H = I_{k} $.
\begin{proposition}
  The solution of the spectral minimum $ NCut $ is the first $ k $ eigenvectors of $ \mathcal{L}_{rw} $ normalized.
\end{proposition}

\end{document}
