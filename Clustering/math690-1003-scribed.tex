\documentclass[12pt]{article}

\usepackage[a4paper,margin=2cm]{geometry}
\usepackage{amsmath, amssymb, amsthm, amsfonts, tikz, algpseudocode}
\usepackage[plain]{algorithm}
\usepackage[framemethod=default]{mdframed}

\theoremstyle{plain}
\newtheorem*{theorem}{Theorem}
\newtheorem*{lemma}{Lemma}
\newtheorem*{claim}{Claim}
\newtheorem*{definition}{Definition}
\newtheorem*{corollary}{Corollary}
\DeclareMathOperator*{\argmin}{arg\,min}
\DeclareMathOperator*{\eig}{eig}

%%%% TITLE

\title{Math 690: Topics in Data Analysis and Computation \\
Lecture notes for October 3, 2017}
\date{}

\author{Scribed by Dev Dabke and Andrew Cho}

\begin{document}
\maketitle

\section{Background}
\label{sec:background}

\subsection{k means}
\label{subsec:kmeans}

Given some data points as input, $ \{ x_i \}^n_{i = 1} $, we want to search for $ k $ clusters $ C = \{ C_1, \ldots, C_k \} $ with ``centers'' $ \mu = \{ \mu_1, \ldots, \mu_k \} $.
\\ \\
Depending on our constraints, we can have different interpretations of what these clusters represent.
\begin{align*}
  \min_{C, \mu} & \sum_{l = 1}^k \sum_{i \in C_{l}} \| x_i - \mu_{l} \|_2^2 & \text{k means} \\
  \min_{C, \mu} & \sum_{l = 1}^k \sum_{i \in C_{l}} \| x_i - \mu_{l} \|_{+} & \text{k medians} \\
  \min_{C, \mu} & \sum_{l = 1}^k \sum_{i \in C_{l}} \| x_i - \mu_{l} \|_2 & \text{$ L_2 $ - $ L_1 $ norm}
\end{align*}
We can choose different initial seeds
\begin{enumerate}
  \item random
  \item by the singular value decomposition (svd)
\end{enumerate}
We are interested in knowing if these results are \textit{consistent}.
Can random seeding lead to the true centroids?

\subsection{Strong Law of Large Numbers}
\label{subsec:slln}

\[
\mathcal{L}_n (\mu) = \int \| x - \mu \|^2 d P_n(x)
\]
where
\[
d P_n (x) = \frac{1}{n} \sum_{i = 1}^n \delta_{x_{i} (x)} dx
\]
which is the empirical mesure of the data.
\\ \\
Where $ A = \{ a_1, \ldots, a_n \} $ for all $ x \in \mathbb{R} $, we define $ \| x - A \| $ to be
\[
\| x - A \| = \min_{1 \leq i \leq n} \| x - a_i \|
\]
Now, note that as $ n \to \infty $ that
\[
\mathcal{L} (\mu) = \int \| x - \mu \|^2 dP(x)
\]
and that
\[
\min_{\mu} \mathcal{L}_n (\mu) \to \min_{\mu} \mathcal{L} (\mu)
\]

\section{Classification and Misclassification}
\label{sec:classmisclass}

\subsection{Analysis}
\label{subsec:analysis}

If we satisfy some requirements of ``separation'' to provide some band for misclassification, assuming some true centroid partition.
\\ \\
$ \| \mu^{*}_{k} - \mu^{*}_{l} \| $ if $ k \neq l $ needs to be bigger than a gap $ \Delta_{kl} $.
\\ \\
\[
\Delta_{kl} > c \frac{\sigma_l}{\sqrt{w_{min}}} \hspace{1cm} w_{min} = \min_{l} \frac{|C_l^{*}|}{n}
\]
Letting $ \sigma_k^2 $ be the variance of the data in the $ k^{th} $ cluster and if we can assume that $ \sigma_k \approx \sigma_l $, then we know that our misclassification error occurs less than some $ \epsilon * n $ with high probability.

\subsection{In Practice}
\label{subsec:inpractice}

There are some practical considerations for k-means.
\begin{enumerate}
  \item what is $ k $?
  \item how to choose an initial seed
  \item some cases of k-means fails
  \begin{enumerate}
    \item $ \sigma_k $ might be too large compared to separation
    \item clusters might be too small (i.e.\ cluster sizes may not be balanced)
    \item cluster might be convex and piecewise can't do concave (Voronoi)
  \end{enumerate}
\end{enumerate}

\section{Spectral Clustering}
\label{sec:spectral}

Given some data $ \{ x_i \}_{i = 1}^n $ and $ k $, we want to
\begin{enumerate}
  \item Build positive-definite, symmetrix matrix $ W_{n \times n} $ by some $ k $ nearest neighborhoods or some $ \epsilon $ neighborhoods.
  Example: $ W_{ij} = e^{- \frac{\| x_i - x_j \|^2}{\epsilon}} $.
  \item Consider the eigenvalue decomposition of the graph Laplacian $ \mathcal{L} $.
  Note that
  \begin{align*}
    L_{wn} &= D - W \\
    L_{rw} &= D^{-1} (D - W) -
  \end{align*}
\end{enumerate}

\end{document}
